\documentclass{mwhittaker}
\title{\fluentl}
\author{\ }
\date{\ }

\usepackage{listings}
\usepackage{pervasives}

\newcommand{\join}{\sqcup}
\newcommand{\fluentl}{Fluent$^L$}

\lstset{ %
  basicstyle=\linespread{0.8}\ttfamily,
  mathescape=true
}


\begin{document}
\maketitle

\section*{Lattices}
A \textbf{partially ordered set} $(S, \leq)$ is a set $S$ together with a
relation $\leq$ on $S$ that is \textit{reflexive}, \textit{antisymmetric}, and
\textit{transitive}. An element $c \in S$ is an \textbf{upper bound} of $a, b
\in S$ if $a \leq c$ and $b \leq c$. An element $c \in S$ is a \textbf{least
upper bound} of $a, b \in S$ if $c$ is an upper bound of $a$ and $b$ and for
any other upper bound $d$ of $a$ and $b$, $c \leq d$.

A \textbf{join semilattice} is a partially ordered set $(S, \leq)$ in which
every pair of elements $a, b \in S$ has a least upper bound. We write this
least upper bound as $a \join b$, read ``$a$ join $b$''. For example, the
semilattices $(2^A, \subseteq)$, $(2^A, \supseteq)$, $(\ints, \leq)$, $(\ints,
\geq)$, $(\bools, \Rightarrow)$, and $(\bools, \Leftarrow)$ have join operators
$\cup$, $\cap$, $\max$, $\min$, $\lor$, and $\land$ respectively. Note that
$\leq$ induces $\join$ and that $\join$ is \textit{associative},
\textit{commutative}, and \textit{idempotent}.
%
We can equivalently define a \textbf{join semilattice} $(S, \join)$ as a set
$S$ together with an associative, commutative, and idempotent binary operator
$\join$. Now, $\join$ induces a partial order $\leq$ in which $a \leq b \iff a
\join b = b$.

Given two join semilattices $(S_1, \leq_1)$ and $(S_2, \leq_2)$, we can
construct the \textbf{product semilattice} $(S_1 \times S_2, \leq)$ where
$(a_1, a_2) \leq (b_1, b_2) \iff a_1 \leq_1 b_1 \land a_2 \leq_2 b_2$.
Similarly, given two join semilattices $(S_1, \join_1)$ and $(S_2, \join_2)$,
we can construct the product semilattice $(S_1 \times S_2, \join)$ where $(a_1,
a_2) \join (b_1, b_2) = (a_1 \join_1 b_1, a_2 \join_2 b_2)$.

\section*{\fluentl}
A \fluentl{} program consists of a sequence of \textbf{variable declarations}
and a sequence of \textbf{rules}. The declarations declare and initialize a set
$x_1, \ldots, x_m$ of variables where $x_i$ stores a value from a join
semilattice $(S_i, \leq_i)$. After the declarations come the rules $r_1,
\ldots, r_n$ where $r_k$ is of the form $x_j \gets m_k(x_j, g_k(\bar{x}))$
where $x_j$ is one of $x_1, \ldots, x_m$; $\bar{x} = x_{i_1}, \ldots, x_{i_l}$
is some subsequence of $x_1, \ldots, x_m$; $g$ is a function of type $S_{i_1}
\times \ldots \times S_{i_l} \to S$ for some join semilattice $(S, \leq)$; and
$m$ is a function (called a \textbf{method}) of type $S_j \times S \to S_j$.
For example, consider the following \fluentl{} program where $g_1(r, s) = r -
s$, $m_1(r, r') = r \cup r'$, $g_2(r) = |r|$, and $m_2(x, x') = x'$.

\begin{lstlisting}
  val $r$: $(2^{\ints \times \ints}, \subseteq)$
  val $s$: $(2^{\ints \times \ints}, \subseteq)$
  val $x$: $(\ints, \leq)$
  $r \gets r \cup (r - s)$
  $x \gets |r|$
\end{lstlisting}

\section*{\fluentl{} Properties}
We say a function $f: S \to S'$ from one partially ordered set to another is
\textbf{monotonic} if for all $a, b \in S$, $a \leq b \implies f(a) \leq f(b)$.
%
We say a function $f: S \to S$ from a partially ordered set to itself is
\textbf{increasing} if for all $a \in S$, $a \leq f(a)$.
%
We similarly say a method $m: (S \times S') \to S$ is increasing if for all $a
\in S$ and for all $a' \in S'$, $a \leq m(a, a')$.

Let $\bar{S}$ be the product semilattice formed from $S_1, \ldots, S_m$. Then,
we can view a \fluentl{} program as a function $f: \bar{S} \to \bar{S}$. Doing
so, we can ask whether $f$ is increasing and whether it is monotonic.
\clmref{monotonic-to-increasing} and \clmref{increasing-to-monotonic} show us
that neither monotonicity nor increasingness implies the other, so we'll have
to consider quite a few cases:

\section*{Proofs}
\begin{claim}\label{clm:monotonic-to-increasing}
  If $f: S \to S$ is monotonic, it may not be increasing.
\end{claim}
\begin{proof}
  Let $f: \ints \to \ints$ be a function between the integers ordered by $\leq$
  where $f(x) = 0$. $f$ is monotonic because for all $a, b \in \ints$, $f(a) =
  0 \leq 0 = f(b)$. However, $f$ is not increasing because $f(1) = 0$ but $1 >
  0$.
\end{proof}

\begin{claim}\label{clm:increasing-to-monotonic}
  If $f: S \to S$ is increasing, it may not be monotonic.
\end{claim}
\begin{proof}
  Let $f: \set{1,2,3} \to \set{1,2,3}$ be a function between the integers
  $\set{1, 2, 3}$ ordered by $\leq$ where $f(1) = 3$, $f(2) = 2$ and $f(3) =
  3$. $f$ is increasing because $1 \leq 3 = f(3)$, $2 \leq 2 = f(2)$, and $3
  \leq 3 = f(3)$. However, $f$ is not monotonic because $1 \leq 2$ but $f(1) =
  3 > 2 = f(2)$.
\end{proof}
\end{document}
