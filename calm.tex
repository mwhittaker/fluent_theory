\documentclass{mwhittaker}
\title{CALM}
\author{\ }
\date{\ }

\usepackage{listings}
\usepackage{pervasives}

\newcommand{\join}{\sqcup}
\newcommand{\fluentl}{Fluent$^L$}

\lstset{ %
  basicstyle=\linespread{0.8}\ttfamily,
  mathescape=true
}

\begin{document}
\maketitle

Consider a program whose state is drawn from some domain $S$. We can model the
program as a function $f: S \to S$ which transforms the input state to some
output state. Now imagine we repeatedly receive a set of states $a, b, c,
\ldots$ over the network. Whenever we receive a new state, we merge it into the
current state using some merge function $\join$, then run $f$ to update our
state. For example, if we started off in state $a$ and received states $b$ and
$c$ over the network, the execution of our program would look something like
this:
\[
  a \to f(a)
    \to f(a) \join b
    \to f(f(a) \join b)
    \to f(f(a) \join b) \join c
    \to f(f(f(a) \join b) \join c)
\]

Similarly, if we had instead started off in state $c$ and then received states
$b$ and $a$ over the network, our execution would look like this:
\[
  c \to f(c)
    \to f(c) \join b
    \to f(f(c) \join b)
    \to f(f(c) \join b) \join a
    \to f(f(f(c) \join b) \join a)
\]

Say a function $f$ is \textbf{order-insensitive} if it is confluent (e.g.
$f(f(f(a) \join b) \join c) = f(f(f(c) \join b) \join a)$) and if all
intermediate states are less than the final state (e.g.  $f(a) \leq f(f(f(a)
\join b) \join c)$ and $f(f(a) \join b) \leq f(f(f(a) \join b))$).

What properties of $f$ and $\join$ are sufficient to ensure
order-insensitivity?

\begin{claim}
  Increasingness is insufficient.
\end{claim}
\begin{proof}
  Let $A = \nats$, $a = \set{0}$, $b = \set{0, 1}$, and \[
    f(x) = \begin{cases}
      \set{0, 1} & x = \set{0} \\
      \set{0, 1, 2} & x = \set{0, 1} \\
      \set{0, 1, 2, 3} & x = \set{0, 1, 2} \\
      x & \text{otherwise} \\
    \end{cases}
  \]
  Clearly $f$ is increasing, but
  \begin{gather*}
    f(f(a) \cup b)
      = f(f(\set{0}) \cup \set{0, 1})
      = f(\set{0, 1} \cup \set{0, 1})
      = f(\set{0, 1})
      = \set{0, 1, 2} \\
    f(f(b) \cup a)
      = f(f(\set{0, 1}) \cup \set{0})
      = f(\set{0, 1, 2} \cup \set{0})
      = f(\set{0, 1, 2})
      = \set{0, 1, 2, 3}
  \end{gather*}
\end{proof}

\begin{claim}
  Monotonicity is insufficient.
\end{claim}
\begin{proof}
  Let $S = \nats$, $a = \set{0, 1}$, and $b = \set{0, 1, 2}$. Given a
  non-empty set $x$, let $\max(x)$ be the largest element in $x$. Let \[
    f(x) = \begin{cases}
      \emptyset & x = \emptyset \\
      x - \set{\max(x)} & x \neq \emptyset
    \end{cases}
  \]
  Next, we show that $f$ is monotonic. Consider arbitrary $x, y \subseteq
  \nats$ such that $x \subseteq y$.
  \begin{itemize}
    \item Case $x = \emptyset, y \neq \emptyset$.
      $f(x) = \emptyset$ which is a subset of all other sets, including
      $f(y)$.

    \item Case $x \neq \emptyset, y \neq \emptyset$.
      If $\max(x) = \max(y)$, then clearly $x - \set{\max(x)} \subseteq y -
      \set{\max(y)}$. Otherwise, $\max(x) \neq \max(y)$. In this case,
      $\max(y)$ cannot be smaller than $\max(x)$ because $\max(x) \in y$.
      Thus, $\max(y) \geq \max(x)$ and therefore $\max(y) \notin x$. Thus, $x
      - \set{\max(x)} \subseteq y - \set{\max(y)}$.

    \item Case $y = \emptyset$.
      Because $x \subseteq y$, $x$ must also be the empty set. Then, $f(x) =
      \emptyset \subseteq \emptyset = f(y)$.
  \end{itemize}
  Finally,
  \begin{gather*}
    f(f(a) \join b)
      = f(f(\set{0, 1}) \join \set{0, 1, 2})
      = f(\set{0} \join \set{0, 1, 2})
      = f(\set{0, 1, 2})
      = \set{0, 1} \\
      f(f(b) \join a)
      = f(f(\set{0, 1, 2}) \join \set{0, 1})
      = f(\set{0, 1} \join \set{0, 1})
      = f(\set{0, 1})
      = \set{0}
  \end{gather*}
\end{proof}

\begin{claim}
  The following properties are sufficient:
  \begin{itemize}
    \item
      Monotonicity: $a \leq b \implies f(a) \leq f(b)$
    \item
      Idempotence: $f(f(x)) = f(x)$
    \item
      $\join$ is associative and commutative, $a \leq a \join b$, and $b \leq a
      \join b$
    \item
      $f(f(a) \join b) = f(f(a \join b))$
  \end{itemize}
\end{claim}
\begin{proof}
  Here's a proof by example that's easy to formalize more carefully. First, we
  show that $f$ is confluent.
  \begin{align*}
    f(f(f(a) \join b) \join c)
      &= f(f(f(a \join b)) \join c) \\
      &= f(f(f(a \join b \join c))) \\
      &= f(f(f(c \join b \join a))) \\
      &= f(f(f(c \join b)) \join a) \\
      &= f(f(f(c) \join b) \join a) \\
  \end{align*}

  Next, we show that all intermediate results are less than the final result.
  \begin{gather*}
    f(a) \to f(f(a) \join b) \to f(f(f(a) \join b) \join c) \\
    f(a) \to f(f(a \join b)) \to f(f(f(a \join b \join c))) \\
    f(a) \to f(a \join b) \to f(a \join b \join c) \\
    f(a) \subseteq f(a \join b) \subseteq f(a \join b \join c)
  \end{gather*}
\end{proof}

Monotonic datalog programs satisfy all the conditions above.
\end{document}
